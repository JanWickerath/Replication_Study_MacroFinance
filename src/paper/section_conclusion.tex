\section{Conclusion}
\label{sec:conclusion}

%example of an introduction which is approximately half a page

This paper reviews “Estimating a Coordination Game in the Classroom” by Petra E. Todd and Kenneth I. Wolpin. Todd and Wolpin attempt to identify factors that lead to students in Mexican schools performing poorly on standardised tests. They use data from the Aligning Learning Incentives (ALI) program. The ALI experiment was run by the Mexican government over a three- year period. It was designed to explore the effect of performance-based monetary incentives for students and teachers on end-of-year mathematics test scores. In the experiment, participating schools were randomly assigned to either one of three treatment groups or a control group.
Analysing the program impact, Behrman et al. (forthcoming) find large treatment effects for some groups. However, both the control and all treatment groups achieve less than 50\% on the curriculum-based examination. To explain these findings, Todd and Wolpin build and estimate a game theoretic model of the effort decisions of teachers and students. 
% Their results suggest that test scores are primarily determined by students’ initial knowledge at the beginning of the academic year.
