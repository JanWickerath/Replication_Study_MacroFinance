\section{Baseline model}
\label{sec:baseline_model}

The following section will shortly discuss the most important differences of
the baseline model in \textcite{jerman_macroeconomic_2012} to the canonical
real business cycle model. After that I will present how I created the shock
processes and discuss differences to the procedure from
\citeauthor{jerman_macroeconomic_2012}. I will then present a short comparison
of the model simulation to the observed data.


The baseline model differs from the canonical real business cycle model
primarily on the firm side of the economy. In this model firms have two sources
of financing: debt and equity. By assumption debt has a tax advantage over
equity and is therefore the preferred source of funding. A similar structure of
firm financing is proposed by \textcite{hennessy_debt_2005}. Firms are allowed
to default on their debt obligations and the decision to default arises
endogenously in the model. As a result the amount of debt a firm can hold will
be limited by the expected liquidation value of its physical capital. The
intuition behind this is that the physical capital will serve as collateral for
the loan. So if the firm decides to default on its debt obligations the lenders
will be able to liquidate the capital but cannot recover any funds from the
liquidity that the firm holds. Anticipating that the firm owner can ``run
away'' with the liquid assets, the lender will only give as much credit as he
can expect to recover in case of a default. The most important difference to
other papers that also study the effects of collateralization in macroeconomic
models (see e.g. \textcite{kiyotaki_credit_1997} and
\textcite{quadrini_financial_2011} for a summary) is the fact that the
liquidation value in \textcite{jerman_macroeconomic_2012} is stochastic. As a
result the financial friction in this model not only amplifies shocks that
originate in the real sector of the economy. The stochastic factor that
determines the liquidation value of physical capital can be interpreted as a
shock that originates in the financial sector.

\subsection{Shock construction}
\label{sec:shock_construction}

\begin{figure}[h]
  \centering
  \includegraphics[width=0.99\textwidth]{../../../bld/out/figures/figure2.pdf}
  \caption{\textit{Time Series of Shocks to Productivity and Financial Conditions}}
  \label{fig:figure1_update}
\end{figure}
\blindtext
\begin{equation}
    A =
    \begin{pmatrix}
        \input{../../../bld/out/tables/ar_matrix.tex}
    \end{pmatrix}
\end{equation}

\subsection{Findings}
\label{sec:findings}

\begin{figure}[h]
  \centering
  \includegraphics[width=.99\textwidth]{../../../bld/out/figures/figure5.pdf}
  \caption{\textit{Response to both Productivity and Financial Shocks}}
  \label{fig:figure5_update}
\end{figure}
\blindtext