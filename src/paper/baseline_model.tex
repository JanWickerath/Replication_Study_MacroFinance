\section{Baseline model}
\label{sec:baseline_model}

The baseline model differs from the canonical real business cycle model
primarily on the firm side of the economy. In this model firms have two sources
of financing: debt and equity. By assumption debt has a tax advantage over
equity and is therefore the preferred source of funding. A similar structure of
firm financing is proposed by \textcite{hennessy_debt_2005}. Firms are allowed
to default on their debt obligations and the decision to default arises
endogenously in the model. As a result the amount of debt a firm can hold will
be limited by the expected liquidation value of its physical capital. The
intuition behind this is that the physical capital will serve as collateral for
the loan. So if the firm decides to default on its debt obligations the lenders
will be able to liquidate the capital but cannot recover any funds from the
liquidity that the firm holds. Anticipating that the firm owner can ``run
away'' with the liquid assets, the lender will only give as much credit as he
can expect to recover in case of a default. The most important difference to
other papers that also study the effects of collateralization in macroeconomic
models (see e.g. \textcite{kiyotaki_credit_1997} and
\textcite{quadrini_financial_2011} for a summary) is the fact that the
liquidation value in \textcite{jerman_macroeconomic_2012} is stochastic. As a
result the financial friction in this model not only amplifies shocks that
originate in the real sector of the economy. The stochastic factor that
determines the liquidation value of physical capital can be interpreted as a
shock that originates in the financial sector.

\subsection{Shock construction}
\label{sec:shock_construction}

Capital is calculated as described in the online appendix:
  \begin{equation}
  \label{eq:capital_original}
  k_{t+1} = k_t - \text{Depreciation} + \text{Investment},
  \end{equation}
where Depreciation is calculated as capital consumption of corporate and
noncorporate business, and investment is calculated as total capital
expenditure of nonfinancial business, both divided by the business price
index.

For Depreciation and Investment I use observations from first quarter of 1952
to the third quarter of 2015. As is done in the original calculation I choose an
initial value for capital such that there is no trend in the ratio of capital
to real business gdp over the period 1952Q1--2015Q3. With two digits precision
this leads to a initial capital stock equal to 22.53. This initial value
describes the capital stock at the beginning of the first period in the
observation sample. With this procedure I get a vector of capital stocks that
starts at the \textbf{beginning of 1952Q1} and ends at the \textbf{end of
  2015Q3}. I remove the initial capital stock from this vector so that the
vector of capital stocks has the same length as my other observed variables and
contains the end-of-period capital stocks from the first quarter of 1952 to the
third quarter of 1952.

Proportional deviations of capital are then obtained by taking the logarithm of
the end-of-period capital series and linearly detrending it over the subperiod
1984Q1--2015Q3. 

Debt levels are also calculated as described in the online appendix:
\begin{equation}
    b_{t+1}^e = b_t^e + \text{Net New Borrowing},
\end{equation}
where Net New Borrowing is taken from the debt securities and loans series for
nonfinancial business as described above. Note that \( b^{e}_{t+1} \) denotes the
debt level at the end of period \(t\).

The initial value for debt, i.e.\ the debt level that is around at the
\textbf{beginning of 1952Q1}, is chosen as in the original program
(94.12). Real debt levels are then computed by dividing debt levels from the
end of 1952Q1 to the end of 2015Q3 by the business price level of the same
period. Proportional deviations are again computed by linearly detrending the
logarithmic series for the period 1984Q1--2015Q3.

Real business gdp is calculated by dividing the series for business value added
by the business price index. Real gdp and the index for working hours are
already normalized. Proportional deviations for all these variables are
calculated as described above.

The index of aggregate working hours data that I use here are slightly
different then the ones used in the original paper. This might be due to a
different normalization of the index, or to data revisions. However, since the
description of how the data series for labor is constructed originally is not
very precise it is difficult to confirm where the differences come from.

The financial states \( \xi \) are calculated by linearizing the enforcement
constraint:
\begin{equation}
\label{eq:fin_shock_correct_timing}
    \hat{\xi}_t = \phi_k \hat{k}_{t+1} + \phi_b \hat{b^e}_{t+1} + \hat{y}_t,
\end{equation}
where \(\phi_k = - \bar{\xi} \frac{\bar{k}}{\bar{y}} = -1.5492\) and \(\phi_b = \bar{\xi} \frac{\bar{b^e}}{\bar{y}} = 0.5588\).
This is equivalent to equation (1) in the online appendix of
\textcite{jerman_macroeconomic_2012}. Here real business gdp (defined above) is
used for the variable \(\hat{y}_t\).

The productivity states are computed from the linearized production function:
\begin{equation}
\label{eq:lin_production}
    \hat{z}_t = \hat{y}_t - \theta \hat{k}_t - (1 - \theta) \hat{n}_t.
\end{equation}
Here I deviate somewhat from the original procedure of
\citeauthor{jerman_macroeconomic_2012}. First I use real business gdp instead
of real total gdp for \(\hat{y}_t\). I decided to do so because in the online
Appendix in the section ``Construction of Productivity Shocks'' they claim that
they apply the same procedure for output as they did when calculating the
financial states. However, in their calculations they acutally use real
business gdp for calculating financial states and real total gdp for
calculating productivity states. The second deviation from the original
procedure is in the timing of equation \ref{eq:lin_production}. Following the
notation from the original paper \(\hat{k}_t\) denotes the the proportional
deviation of capital at the end of period \(t - 1\). However, in the original
program \(\hat{z}_t\) is calculated as a function of \(\hat{k}_{t +
  1}\). Obviously this does not make sense economically, so I stick to the
timing convention from equation \ref{eq:lin_production}.

Note that I only have proportional deviations of capital stocks for the end of
period 1984Q1 onwards. Hence the vector of productivity values ranges from the
\textbf{second quarter of 1984 to the third quarter of 2015}. Since both shock
vectors have to be of the same length, I also only calculate the financial
shocks for this time span.

\begin{figure}[h]
  \centering
  \includegraphics[width=0.99\textwidth]{../../../bld/out/figures/figure2.pdf}
  \caption{\textit{Time Series of Shocks to Productivity and Financial Conditions}}
  \label{fig:figure1_update}
\end{figure}

The innovations to the financial state \(\epsilon_{\xi}\) and the productivity
state \(\epsilon_{z}\) are estimated as the residuals from the autoregressive
system in equation (12) of the original paper. Figure \ref{fig:figure2_update}
plots the levels and innovations of both shocks in the model. A problem in this
procedure is that the resulting shock innovations are highly
correlated. Calculating a correlation coefficient between \(\epsilon_{\xi}\)
and \(\epsilon_z\) gives a result of $0.6895$. This result is
quite problematic since it is at odds with the author's interpretation of the
two innovation processes as independent shocks. Empirically the model is not
able to clearly distinguish the shocks.

% \begin{equation}
%     A =
%     \begin{pmatrix}
%         \input{../../../bld/out/tables/ar_matrix.tex}
%     \end{pmatrix}
% \end{equation}

\subsection{Findings}
\label{sec:findings}

Figure \ref{fig:figure5_update} plots simulated series for proportional
deviations of output, working hours and financial flows from their respective
steady state values in the baseline model, when both shock series estimated
above are plugged in to the model. It compares these simulated data to its
empirical counterparts. 

\begin{figure}[h]
  \centering
  \includegraphics[width=.99\textwidth]{../../../bld/out/figures/figure5.pdf}
  \caption{\textit{Response to both Productivity and Financial Shocks}}
  \label{fig:figure5_update}
\end{figure}

To make the observed data comparable to proportional deviations from the steady
state I need to make sure that they have no trend and are centred around
zero. To achieve this I follow the procedure from
\textcite{jerman_macroeconomic_2012}. For output I take the logarithm of real
GDP (obtained from the national income and product accounts, see appendix for
further details) and linearly detrend it over the period from first quarter
1984 to the third quarter 2015. The same procedure I apply to the index of
aggregate weekly hours from the current employment statistics (again, see
appendix for details). For equity payouts and debt repurchases and equity
payouts I use the data series constructed in Section \ref{sec:data} and
linearly detrend them over the same period as above.
