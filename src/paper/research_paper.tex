\documentclass[12pt,a4paper,leqno]{article}
\usepackage{a4wide}
\usepackage[T1]{fontenc}
\usepackage[utf8]{inputenc}
\usepackage{float, afterpage, rotating, graphicx}
\usepackage{longtable, booktabs, tabularx}
\usepackage{verbatim}
\usepackage{eurosym, calc, chngcntr}
\usepackage{amsmath, amssymb, amsfonts, amsthm, bm, delarray} 
\usepackage[font=footnotesize, labelfont=sc, justification=centering]{caption}
\usepackage{tkz-graph}
\usepackage{graphicx}
\usepackage{subfig}
\usepackage{float}
\usepackage{natbib} 
\usepackage{setspace}
\usepackage{titlesec}

\restylefloat{table}
% \usetikzlibrary{arrows,positioning,snakes,shapes,shapes.multipart,patterns,mindmap,shadows}

\usepackage[unicode=true]{hyperref}
\hypersetup{
    colorlinks=true,
    linkcolor=black,
    anchorcolor=black,
    citecolor=black,
    filecolor=black,
    menucolor=black,
    runcolor=black,
    urlcolor=black
}


\widowpenalty=10000
\clubpenalty=10000

\setlength{\parskip}{1ex}
\setlength{\parindent}{0ex}
\setstretch{1.5}


\begin{document}

\title{Macroeconomic Effects of Financial Shocks: \\A Replication Study
\thanks{Nikolas Kuhlen: University of Bonn. \href{mailto:nikolas.kuhlen@uni-bonn.de} {\nolinkurl{nikolas.kuhlen [at] uni-bonn [dot] de}}.}
% subtitle: 
% \\[1ex] 
% \large Subtitle here
}

\author{Nikolas Kuhlen
% \\[1ex]
% Additional authors here
}

\date{\today}

\maketitle


% \begin{abstract}
% 	Some abstract here.
% \end{abstract}
\clearpage

\section{Introduction}
\label{sec:introduction}

\newpage
\section{Data}
\label{sec:data}

% \begin{figure}[h]
% \centering{
% % \subfloat[Financial Flows, 1952:I--2010:II]{{\includegraphics[scale=0.39]{Replication_Figure_1.pdf} }}
% \subfloat[Financial Flows, 1952:I--2015:II]{{\includegraphics[scale=0.39]{../../bld/out/figures/Replication_Figure_1_updated.pdf} }}
% }
% \end{figure}

\begin{figure}
    
    \centering

    \includegraphics[scale=0.65]{../../out/figures/Replication_Figure_1_updated}

    \caption{\textsc{Financial Flows in the Nonfinancial Business Sector \textit{(Corporate and Noncorporate)}}, 1952:I--2015:II}
    
    \label{fig:figure_1}

\end{figure}


Figure \ref{fig:figure_1} illustrates the financial cycles in the US economy. For this purpose, it plots net payments to equity holders and net debt repurchases in the nonfinancial business sector. The data is taken from the Flow of Funds Accounts (FFA) of the Federal Reserve Board (FRB).\footnote{Link to original FFA data: \href{http://www.federalreserve.gov/datadownload/Download.aspx?rel=Z1&series=1f08e962a27dff21b89a7212d58b8346&filetype=spreadsheetml&label=include&layout=seriescolumn&from=03/01/1952&to=06/30/2015}{here}; last checked: January 28, 2016.}
References to the FFA are indicated by making use of the ‘Coded Tables’ published on September 18, 2015. The current version of the FFA has undergone some modifications since \citet{JERMANNfinancial} was published. As a consequence, some of the identifiers have been replaced in the current version of the FFA. Table \ref{table:identifiers} in the Appendix contains the relevant changes.

Figure \ref{fig:figure_1} is using the updated FFA data released in December 2015 containing observations from the first quarter of 1952 to the second quarter of 2015. Equity Payout is obtained as described by \citeauthor{JERMANNfinancial}. First, the sum of ‘Net dividends of nonfarm, nonfinancial business’ (FA106121075.Q) and ‘Net dividends of farm business’ (FA136121073.Q) is being calculated. This is followed by subtracting ‘Net increase in corporate equities of nonfinancial business’ (FA103164103.Q) and ‘Proprietors’ net investment of nonfinancial business’ (FA112090205.Q). Debt Repurchase in the figure above is the negative of ‘Net increase in credit markets instruments of nonfinancial business’ (FA144104005.Q). This implies that debt consists only of liabilities that are directly related to credit markets transactions. Both variables are divided by Business GDP times 1000. Business GDP is based on data taken from the National Income and Product Accounts (Table 1.3.5) on Business Value Added for the period 1952:I--2015:II.\footnote{Link to original NIPA data: \href{http://www.bea.gov//national/nipaweb/DownSS2.asp}{here}; last checked: November 7, 2015.} 

The figure based on the updated data generally indicates the same negative correlation between equity payout and debt repurchases. In particular, the strong overall relationship between the two variables as described by \citeauthor{JERMANNfinancial} can be replicated for period after 1980. This also holds true for the new values starting from the third quarter of 2010. Consequently, the replication supports the hypotheses by \citeauthor{JERMANNfinancial} which claim a certain degree of substitutability between equity and debt financing as well as recessions inducing firms to decrease in the growth rate of debt while at the same time lowering payments to shareholders. However, for the period from 1956 to 1980 as well as the period after 1992 especially the series on debt repurchases looks more volatile when using the updated data. The reason for this difference are revisions of the series for those periods in the most recent FFA data by the FRB.

\section{Baseline Model}
\label{sec:baseline_model}

\newpage
\section{Shock Construction}
\label{sec:shock_construction}

\section{Findings}
\label{sec:findings}

\section{Extended Model}
\label{sec:extended_model}

\section{Estimation of the Extended Model}
\label{sec:estimation_extended_model}

\section{Sensitivity Analysis}
\label{sec:sensitivity}

\newpage
\vfill
\setstretch{1}
\addcontentsline{toc}{section}{References}
\bibliographystyle{ecta}
\bibliography{refs}

\setstretch{1.5}

\newpage
\section*{Appendix}
\addcontentsline{toc}{section}{Appendix}
\setcounter{table}{0}
\renewcommand{\thetable}{A\arabic{table}}
% \counterwithin{table}{section}
% \setcounter{table}{0}
%\counterwithin{figure}{section}

% \subsection*{Data}
As described in section \ref{sec:data}, certain codes for the FFA data have changed compared to \citeauthor{JERMANNfinancial}. Table \ref{table:identifiers} contains all relevant code changes.\footnote{For future reference, all changes are listed: \href{http://www.federalreserve.gov/apps/fof/CodeChange.aspx}{here}; last checked: January 30, 2016.} 

\begin{table}[h]
\small
\caption{Code Changes}
\centering{
\begin{tabular}{l|c|c}

\hline\hline
\bf{Description} & \bf{Old Identifier} & \bf{New Identifier}\\[1ex]
\hline 
Net dividends of nonfarm, nonfinancial business & FA106120005.Q &  FA106121075.Q\\[0.2ex]
Net dividends of farm business & FA136120005.Q & FA136121073.Q\\[0.2ex]
Proprietors’ net investment of nonfinancial business & FA142090205.Q & FA112090205.Q\\[1ex]
\hline\hline

\end{tabular}
}\\
\label{table:identifiers}
\end{table}


\end{document}
