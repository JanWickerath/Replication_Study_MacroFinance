\documentclass[11pt,a4paper,leqno]{article}
\usepackage{a4wide}
\usepackage[T1]{fontenc}
\usepackage[utf8]{inputenc}
\usepackage{float, afterpage, rotating, graphicx}
\usepackage{longtable, booktabs, tabularx}
\usepackage{verbatim}
\usepackage{eurosym, calc, chngcntr}
\usepackage{amsmath, amssymb, amsfonts, amsthm, bm, delarray} 
\usepackage{caption}
\usepackage{tkz-graph}
\usepackage{graphicx}
\usepackage{subfig}
\usepackage{float}
\usepackage{natbib} 
\usepackage{setspace}
\usepackage{titlesec}

\restylefloat{table}
% \usetikzlibrary{arrows,positioning,snakes,shapes,shapes.multipart,patterns,mindmap,shadows}

\usepackage[unicode=true]{hyperref}
\hypersetup{
    colorlinks=true,
    linkcolor=black,
    anchorcolor=black,
    citecolor=black,
    filecolor=black,
    menucolor=black,
    runcolor=black,
    urlcolor=black
}


\widowpenalty=10000
\clubpenalty=10000

\setlength{\parskip}{1ex}
\setlength{\parindent}{0ex}
\setstretch{1.5}


\begin{document}

\title{Macroeconomic Effects of Financial Shocks: A Replication Study
\thanks{Nikolas Kuhlen: University of Bonn, Address. \href{mailto:x@y.z} {\nolinkurl{x [at] y [dot] z}}, tel.~+00000.}
% subtitle: 
% \\[1ex] 
% \large Subtitle here
}

\author{Nikolas Kuhlen
% \\[1ex]
% Additional authors here
}

\date{\today}

\maketitle


\begin{abstract}
	Some abstract here.
\end{abstract}
\clearpage

\section{Introduction} % (fold)
\label{sec:introduction}

\section{Data} % (fold)
\label{sec:data}

\begin{figure}[h]
\centering{
% \subfloat[Financial Flows, 1952:I--2010:II]{{\includegraphics[scale=0.39]{Replication_Figure_1.pdf} }}
\subfloat[Financial Flows, 1952:I--2015:II]{{\includegraphics[scale=0.39]{../../bld/out/figures/Replication_Figure_1_updated.pdf} }}
}
\end{figure}

Both figures (a) and (b) plot the net payments to equity holders and the net debt repurchases in the nonfinancial business sector. The financial data is taken from the Flow of Funds Accounts of the Federal Reserve Board. References to the FFA are indicated by making use of the ‘Coded Tables’ published on September 17, 2010 and September 18, 2015. The current version of the FFA has undergone some modifications since \citet{JERMANNfinancial} was published. As a consequence, some of the identifiers have been changed in the current  version FFA. These changes are specified in the updated data set where applicable.

Figure (a) is based on the data provided by \citeauthor{JERMANNfinancial}. In particular, the data set contains observations from the first quarter of 1952 to the second quarter of 2010. Figure (b) displays the updated series from 2015 containing observations from 1952:I to 2015:II.\footnote{Link to original FFA data: \href{http://www.federalreserve.gov/datadownload/Download.aspx?rel=Z1&series=6215a816d08cd7918c39d61f83656f4d&filetype=spreadsheetml&label=include&layout=seriescolumn&from=03/01/1952&to=06/30/2015}{here}; last checked: November 7, 2015.}

Equity Payout is obtained as follows: first, calculate the sum of ‘Net dividends of nonfarm, nonfinancial business’ (FA106121075.Q) and ‘Net dividends of farm business’ (FA136121073.Q). This is followed by subtracting ‘Net increase in corporate equities of nonfinancial business’ (FA103164103.Q) and ‘Proprietors’ net investment of nonfinancial business’ (FA112090205.Q). Debt consists only of liabilities that are directly related to credit markets transactions.  Debt Repurchase is the negative of ‘Net increase in credit markets instruments of nonfinancial business’ (FA144104005.Q).
Both variables are divided by business GDP. For 1952:I--2010:II, the data on business GDP from \citeauthor{JERMANNfinancial} is being used. Values for the period 1952:I--2015:II are based on the updated business value added from the National Income and Product Accounts (Table 1.3.5).\footnote{Link to original NIPA data: \href{http://www.bea.gov//national/nipaweb/DownSS2.asp}{here}; last checked: November 7, 2015.} 

% section introduction (end)

\vfill
\setstretch{1}
\addcontentsline{toc}{section}{References}
\bibliographystyle{ecta}
\bibliography{refs}

\setstretch{1.5}

%\appendix
%\counterwithin{table}{section}
%\counterwithin{figure}{section}

\end{document}
