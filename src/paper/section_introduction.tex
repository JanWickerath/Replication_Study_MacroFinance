\section{Introduction}
\label{sec:introduction}

This paper documents the replication and update of “Macroeconomic Effects of Financial Shocks” by Urban Jermann and Vincenzo Quadrini. \citeauthor{JERMANNfinancial} attempt to analyse the effect of shocks originating in the financial sector on business cycle fluctuations. For this purpose, they employ two distinct methodological approaches. First, they implement firms' debt and equity financing in the canonical real business cycle (RBC) model. Using the financial shocks obtained from the augmented model, they simulate the response of real and financial variables. Second, \citeauthor{JERMANNfinancial} build and estimate a structural model based on \citet{SMETSshocks}. This allows them to assess the impact of financial shocks relative to other sources of shocks. Their results suggest that financial shocks have a significant effect on macroeconomic fluctuations. In addition to the economic implications of these findings, the replication of the results is important in its own right. For example, \cite{CHANGeconomics} find that research papers published in macroeconomics and general interest economics journals are most often not replicable. 

The remainder of the paper is organised as follows. The following section empirically examines the financial cycle in the US economy. Section \ref{sec:baseline_model} and \ref{sec:extended_model} discuss the two approaches \citeauthor{JERMANNfinancial} use to investigate the effects of financial shocks. Section \ref{sec:conclusion} concludes.


% The replication and update of the original paper yields results which are similar to the findings by \citeauthor{JERMANNfinancial}. However, the majority of the original results cannot be reproduced exactly. 
% This is particularly true for the baseline model due to inconsistencies between the online appendix and the provided code.
