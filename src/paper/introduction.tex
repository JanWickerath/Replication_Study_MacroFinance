
\section{Introduction}
\label{sec:introduction}

The effects of frictions of financial markets on aggregate economic activity
have been studied for a while now. For example the seminal work of
\textcite{kiyotaki_credit_1997}, \textcite{bernanke_agency_1989} and
\textcite{bernanke_financial_1999} have studied mechanisms that explain how
frictions on financial markets can amplify shocks that originated in the real
sector of the economy. Empirically the availability of credit plays an
important role for determining real economic
activity. \textcite{kindleberger_manias_2011} identify the availability of
credit as a main driver of historical financial crises and
\textcite{schularick_credit_2012} find that the level of financial stability
can often be explained by credit aggregates.

\textcite{jerman_macroeconomic_2012} build a model in which credit availability
is restricted and partially determined by a random factor. They interpret this
random factor as a shock that originates in the financial sector and use their
model to study the effect of this financial shock on real economic variables.

The following term paper will replicate and update the main results from
\textcite{jerman_macroeconomic_2012}. Section \ref{sec:data} presents an update
of the financial flows of firms that provided the main motivation for the
analysis in \textcite{jerman_macroeconomic_2012}. Section
\ref{sec:baseline_model} will shortly discuss the most important differences of
the baseline model in \textcite{jerman_macroeconomic_2012} to the canonical
real business cycle model. After that I will present how I created the shock
processes and discuss differences to the procedure from
\citeauthor{jerman_macroeconomic_2012}. I will then present a short comparison
of the model simulation to the observed data. Section \ref{sec:extended_model}
introduces the main differences from the extended model to the baseline model,
will present a short description of the estimation of the extended model and
then discuss the convergence of the estimation process.