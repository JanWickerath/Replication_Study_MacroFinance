\section{Extended Model}
\label{sec:extended_model}

The extended model of \textcite{jerman_macroeconomic_2012} essentially
adds the financial friction develeoped in the baseline model to the
model from \textcite{smets_shocks_2007}. Compared to the baseline model the
extended model adds nominal rigidities in wages and prices. To do so the model
follows the new keynesian literature and incorporates an explicit wage and
price setting mechanism of monopolistically competitive households and
firms. The wage setting mechanism follows a setting with Calvo's price
rigidity \parencite{calvo_staggered_1983} whereas the price setting mechanism
is rigid due to convex price adjustment costs. Additionally the extended model
includes a public sector. The public sector consists of a fiscal authority, that
makes unproductive government purchases and grants interest deduction to
debtholders and balances its budget by collecting lump-sum taxes from the
households, and a monetary authority that sets the interest rate in the economy
according to some taylor rule.

\subsection{Estimation}
\label{sec:estimation}

For the estimation of the extended model I use the original data from
\textcite{jerman_macroeconomic_2012}. They estimate the model using eight
empirical series for GDP growth, personal consumption growth rate, inflation,
growth rate of working hours, growth rate of hourly wages, federal funds rate
and debt repurchases of nonfinancial businesses. The data are provided in a
form that is not directly comparable to the model equivalents. Hence as in the
baseline model I need to transform them to match the definition of their
respective model variable. For GDP, consumption, investment, price deflator,
working hours and wages I do so by taking log-differences and demean the
resulting series. Since the federal funds rate is already expressed in
percentage terms I demean the series without taking log-differences. I linearly
detrend debt repurchases as this is the same procedure that was taken in the
baseline model for Figure \ref{fig:figure5_update}.\footnote{Note that we can
  see in Figure \ref{fig:figure1_update} that debt repurchases hav no clear
  trend. Hence demeaning the series instead of linearly detrending it does not
  make a remarkable difference.}

\begin{table}
  \centering
  \resizebox{\textwidth}{!}{
    \begin{tabular}{llccc}
      \toprule
      Calibrated parameters & & & & Value\\
      \midrule
      \input{../../../bld/out/tables/cal_params.tex}
      \midrule
      Estimated parameters & Prior [mean, std] & Mode & Below 5\% & Below 95 \% \\
      \midrule
      \input{../../../bld/out/tables/est_params.tex}
      \bottomrule    
    \end{tabular}
  }
  \caption{Parameterization}
  \label{tab:estimation}
\end{table}

Table \ref{tab:estimation} shows the parameter values for the calibrated
parameters. For the estimated shocks it shows the prior distribution as well as
the mode and 95\% HPD-interval of the posterior distribution. The posterior
distribution is reached using the Metropolis-Hastings algorithm implemented in dynare.

\subsection{Convergence of the posterior distribution}
\label{sec:convergence}
\blindtext
\begin{figure}[h]
  \begin{subfigure}{.5\textwidth}
    \centering
    \includegraphics[width=0.99\textwidth]{../../../bld/out/figures/TracePlot_SE_eps_xi.eps}
    \caption{\textit{Traceplot for \(\sigma_{\xi}\)}}
    \label{fig:sig_traceplot}
  \end{subfigure}
  \begin{subfigure}{.5\textwidth}
    \centering
    \includegraphics[width=0.99\textwidth]{../../../bld/out/figures/TracePlot_SE_eps_z.eps}
    \caption{\textit{Traceplot for \(\sigma_{z}\)}}
    \label{fig:tfp_traceplot}
  \end{subfigure}
  \caption{\textit{Traceplots over 3 Mio. iterations of the MCMC-sampler}}
  \label{fig:convergence_eps_xi}
\end{figure}




