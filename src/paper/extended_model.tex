\section{Extended Model}
\label{sec:extended_model}

The extended model of \textcite{jerman_macroeconomic_2012} essentially adds the
financial friction develeoped in the baseline model to the model from
\textcite{smets_shocks_2007}. Compared to the baseline model the extended model
adds nominal rigidities in wages and prices. To do so the model follows the new
keynesian literature and incorporates an explicit wage and price setting
mechanism of monopolistically competitive households and firms. The wage
setting mechanism follows a setting with Calvo's price
rigidity \parencite{calvo_staggered_1983} whereas the price setting mechanism
is rigid due to convex price adjustment
costs \parencite{rotemberg_sticky_1982}. Additionally the extended model
includes a public sector. The public sector consists of a fiscal authority,
that makes unproductive government purchases and grants interest deduction to
debtholders and balances its budget by collecting lump-sum taxes from the
households, and a monetary authority that sets the interest rate in the economy
according to some taylor rule.

\subsection{Estimation}
\label{sec:estimation}

For the estimation of the extended model I use the original data from
\textcite{jerman_macroeconomic_2012}. They estimate the model using eight
empirical series for GDP growth, personal consumption growth rate, inflation,
growth rate of working hours, growth rate of hourly wages, federal funds rate
and debt repurchases of nonfinancial businesses. The data are provided in a
form that is not directly comparable to the model equivalents. Hence as in the
baseline model I need to transform them to match the definition of their
respective model variable. For GDP, consumption, investment, price deflator,
working hours and wages I do so by taking log-differences and demean the
resulting series. Since the federal funds rate is already expressed in
percentage terms I demean the series without taking log-differences. I linearly
detrend debt repurchases as this is the same procedure that was taken in the
baseline model for Figure \ref{fig:figure5_update}.\footnote{Note that we can
  see in Figure \ref{fig:figure1_update} that debt repurchases hav no clear
  trend. Hence demeaning the series instead of linearly detrending it does not
  make a remarkable difference.}

\begin{table}
  \centering
  \caption{Parameterization}
  \label{tab:estimation}
  \resizebox{\textwidth}{!}{
    \begin{tabular}{llccc}
      \toprule
      Calibrated parameters & & & & Value\\
      \midrule
      \input{../../../bld/out/tables/cal_params.tex}
      \midrule
      Estimated parameters & Prior [mean, std] & Mode & Below 5\% & Below 95 \% \\
      \midrule
      \input{../../../bld/out/tables/est_params.tex}
      \bottomrule
    \end{tabular}
  }
  \smallskip

  \raggedright
  {\footnotesize \textit{The prior distributions for the average price
    markup $\bar{\eta}$ and the average wage markup $\bar{\nu}$ are generalized
    beta distributions with support [1, 2] }}
\end{table}

Table \ref{tab:estimation} shows the parameter values for the calibrated
parameters. For the estimated shocks it shows the prior distribution as well as
the mode and 95\% HPD-interval of the posterior distribution. The posterior
distribution is reached using the Metropolis-Hastings algorithm implemented in dynare.

\subsection{Convergence of the posterior distribution}
\label{sec:convergence}

The following section is going to discuss some convergence diagnostics for the
Markov-Chain Monte-Carlo sampler that was used to obtain the posterior
estimates for table \ref{tab:estimation}. For visual inspection of the
posterior draws, figure \ref{fig:convergence_eps_xi} show trace plots for three
million draws from the MCMC-sample algorithm for the estimated variances of the
financial shocks \(\xi\) and the tfp shocks \(z\). For the productivity shock
we can see no trend in the draws as indicated by the almost flat line that
represents the moving average over three hundred thousand sampled values. This
is a good sign that the Markov chain of sampled values has converged to its
stationary distribution.

A more interesting case is the trace plot for the estimated standard deviation
of the financial shock. Here we can, just by visual inspection, find jumps in
the chain of sampled values for more than two million iterations of the
MCMC-sampler. So even at such a large number of iterations it is not clear
whether the chain of sampled values has actually converged to the true
posterior distribution. 

Given the observations above it is remarkable that the original paper from
\textcite{jerman_macroeconomic_2012} provides results based on only ten
thousand draws from the posterior distribution. At this stage even the chain of
draws for the estimated standard deviation of the tfp shocks has not converged.
