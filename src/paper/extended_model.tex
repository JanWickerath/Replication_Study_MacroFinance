\section{Extended Model}
\label{sec:extended_model}

The extended model of \textcite{jerman_macroeconomic_2012} essentially
adds the financial friction develeoped in the baseline model to the
model from \textcite{smets_shocks_2007}. Compared to the baseline model the
extended model adds nominal rigidities in wages and prices. To do so the model
follows the new keynesian literature and incorporates an explicit wage and
price setting mechanism of monopolistically competitive households and
firms. The wage setting mechanism follows a setting with Calvo's price
rigidity \parencite{calvo_staggered_1983} whereas the price setting mechanism
is rigid due to convex price adjustment costs. Additionally the extended model
includes a public sector. The public sector consists of a fiscal authority, that
makes unproductive government purchases and grants interest deduction to
debtholders and balances its budget by collecting lump-sum taxes from the
households, and a monetary authority that sets the interest rate in the economy
according to some taylor rule.

\subsection{Estimation}
\label{sec:estimation}

For the estimation of the extended model I use the original data from
\textcite{jerman_macroeconomic_2012}. 

\begin{table}
  \centering
  \resizebox{\textwidth}{!}{
    \begin{tabular}{llccc}
      \toprule
      Calibrated parameters & & & & Value\\
      \midrule
      \input{../../../bld/out/tables/cal_params.tex}
      \midrule
      Estimated parameters & Prior [mean, std] & Mode & Below 5\% & Below 95 \% \\
      \midrule
      \input{../../../bld/out/tables/est_params.tex}
      \bottomrule    
    \end{tabular}
  }
  \caption{Parameterization}
  \label{tab:estimation}
\end{table}

\subsection{Convergence}
\label{sec:convergence}

\begin{figure}[h]
  \begin{subfigure}{.5\textwidth}
    \centering
    \includegraphics[width=0.99\textwidth]{../../../bld/out/figures/TracePlot_SE_eps_xi.eps}
    \caption{\textit{Traceplot for \(\sigma_{\xi}\)}}
    \label{fig:sig_traceplot}
  \end{subfigure}
  \begin{subfigure}{.5\textwidth}
    \centering
    \includegraphics[width=0.99\textwidth]{../../../bld/out/figures/TracePlot_SE_eps_z.eps}
    \caption{\textit{Traceplot for \(\sigma_{z}\)}}
    \label{fig:tfp_traceplot}
  \end{subfigure}
  \caption{\textit{Traceplots over 3 Mio. iterations of the MCMC-sampler}}
  \label{fig:convergence_eps_xi}
\end{figure}




