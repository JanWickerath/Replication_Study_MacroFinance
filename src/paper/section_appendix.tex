\subsection{Data}
\label{sec:appendix_data}

For a detailed description of the data sets used for the replication please refer to the project documentation.

As described in section \ref{sec:data}, certain codes for the FFA data have changed compared to \citeauthor{JERMANNfinancial}. Table \ref{table:identifiers} contains all relevant code changes.\footnote{For future reference, all changes are listed: \href{http://www.federalreserve.gov/apps/fof/CodeChange.aspx}{here}; last checked: January 30, 2016.} 

% This creates the table containing all relevant code changes in identifiers from the FFA data set.
\begin{table}[h]
\small
\caption{Code Changes}
\centering{
\begin{tabular}{l|c|c}

\hline\hline
\bf{Description} & \bf{Old Identifier} & \bf{New Identifier}\\[1ex]
\hline 
Net dividends of nonfarm, nonfinancial business & FA106120005.Q &  FA106121075.Q\\[0.2ex]
Net dividends of farm business & FA136120005.Q & FA136121073.Q\\[0.2ex]
Proprietors’ net investment of nonfinancial business & FA142090205.Q & FA112090205.Q\\[1ex]
Consumption of fixed capital in nonfinancial & FA106300053.Q & FA106300083.Q \\ 
corporate business & & \\[1ex]
Consumption of fixed capital in nonfinancial & FA116300081.Q & FA116300001.Q\\
noncorporate business & & \\[1ex]
\hline\hline

\end{tabular}
}\\
\label{table:identifiers}
\end{table}



\subsection{Shock Construction}
\label{sec:appendix_shock_construction}

Using the updated data from the FFA, NIPA and Current Employment Statistics from the Bureau of Labor Statistics as well as correcting and adjusting the construction of shocks as described in section \ref{sec:shock_construction} yields the following estimates for matrix $A$:
\begin{center}
\[
    A=
      \begin{bmatrix}
    	\input{../../out/tables/table_matrix_a.txt}
      \end{bmatrix}
\]
\end{center}

The following figure is used in section \ref{sec:shock_construction} to examine the presence of residual autocorrelation:

\begin{figure}[ht]
	\centering{
		\subfloat[Autocorrelation of ${\epsilon}_{z,t+1}$]{{\includegraphics[scale=0.33]{../../out/figures/autocorrelation_tfp.pdf}}}
		\subfloat[Autocorrelation of ${\epsilon}_{\xi,t+1}$]{{\includegraphics[scale=0.33]{../../out/figures/autocorrelation_xi.pdf}}}
	}
	\caption{Residual autocorrelation}
	\label{fig:residual_autocorrelation}
\end{figure}