\begin{figure}[ht]
    
    \centering

    \includegraphics[scale=0.66]{../../out/figures/figure_2}

    \caption{\textsc{Time Series of Shocks to Productivity and Financial Conditions}}
    
    \label{fig:figure_2}

\end{figure}

The replication of figure \ref{fig:figure_2} is based on the updated data from the FFA, NIPA and Current Employment Statistics from the Bureau of Labor Statistics.\footnote{Link to original BLS data: \href{https://research.stlouisfed.org/fred2/series/AWHI/downloaddata}{here}; last checked: November 21, 2015.} 
The procedure employed to construct the shock series generally follows the steps as outlined by \citeauthor{JERMANNfinancial} in the online appendix. However, some steps have been corrected or adjusted in the calculation of the tfp shocks. These deviations will be addressed in the following. 

First, the value for initial capital is set to $22.38$ such that there is no trend in the ratio of capital to real business gdp for the updated capital series. Second, the timing in the construction of the tfp shocks employed by \citeauthor{JERMANNfinancial} contradicts the calculation of the capital stock as specified in the online appendix. Adhering to the timing convention for the capital stock, TFP shocks in the current quarter are calculated using last quarter's end of period capital stock in the replication of figure \ref{fig:figure_2}. 
Third, the online appendix implies that the TFP shock series is calculated using business GDP. The original data set, however, reveals that \citeauthor{JERMANNfinancial} use real GDP to obtain their TFP values. The shocks used to replicate figure \ref{fig:figure_2} are constructed following the description in the online appendix. 

Appendix \ref{sec:appendix_shock_construction} contains the estimates for matrix $A$ based on the updated data. The results indicate that the off-diagonal elements of matrix $A$ are strongly influenced by the corrections described above and in particular by the choice of GDP. 
In addition, Appendix \ref{sec:appendix_shock_construction} contains figure \ref{fig:residual_autocorrelation} to visually examine the presence of residual autocorrelation.
The graph shows that there is significant autocorrelation present in ${\epsilon}_{\xi,t+1}$ when considering the first to third lag. This implies that the residuals do not resemble white noise in the VAR model. This is supported by the fact that there is a correlation of 0.6974 between the residuals. As a consequence, the model might be misspecified.

Following the original figure in Jermann and Quadrini, figure \ref{fig:figure_2} only plots values for the period after 1985.
